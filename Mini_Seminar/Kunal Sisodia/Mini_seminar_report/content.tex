%%%%%%%%%%%%%%%%%%%%%%%%%%%%%%%%%%%%%%
% START ADDING TEXT HERE 
%
% Feel free to use \include commands to structure text in smaller
% pieces 
% 
%%%%%%%%%%%%%%%%%%%%%%%%%%%%%%%%%%%%%%


% Abstract gives a brief summary of the main points of a paper:
\begin{abstract}
\textbf{\textit{Abstract}} - 
\begin{bfseries} The current era we are living in is marked with the evolution of high tech breakthroughs with technology like fifth-generation wireless technologies (5G) which have provided firm grip to the communication networks by increasing the potential and efficiency. For all of its advantages, the upcoming technologies are a double-edged sword which are followed by some drawbacks. Apparently, One prime problem is of Traffic Engineering. To counter this a novel approach DRL-TE was developed which was inspired by Deep Neural Networks (DNNs). TE-aware exploration and actor-critic-based prioritized experience replay techniques were proposed as the part of the framework to boost the performance of DRL framework. ns-3 was deployed for implementation and extensive testing of this technique which later overshadowed all currently used techniques by offering improved utility, reducing delays, better throughput and thus better efficiency.
\end{bfseries}

\end{abstract}

% the actual content, usually separated over a number of sections
% each section is assigned a label, in order to be able to put a
% crossreference to it

\section{Introduction}
\label{sec:introduction}

The main objective of the author was to research and develop a self-reliance framework for the highly advance and dynamic modern communication network. The framework should be capable of controlling the network using its own experience rather instead of some pre-written algorithms. A challenging problem which arises in this domain is to forward the packet from source to destination in the network by maximizing the utility function which is termed as Traffic Engineering Problem. To counter this several techniques like OSPF, VLB, queuing theory and NUM were deployed but neither of them proves to be the optimal and best solution for this problem. Furthermore with the recent AI breakthrough of AlphaGo by Deepmind paved the way to solve the advance complex network problem using deep reinforcement learning.
The DRL \cite{mnih2015humanlevel} Approach gave some significant advantages Firstly this approach was model free which in turn enhanced its applicability.
Secondly DRL technique was able to deal with highly dynamic time-variant environments.
Moreover DRL was capable of handling a sophisticated state space, which is more advantageous over traditional Reinforcement Learning (RL).
There were also some drawbacks associated with DRL Approach - Direct application of basic DRL technique such as DQN \cite{mnih2015humanlevel} based DRL did not worked well for TE prob as DQN was only capable to handle control problems with a limited action space.
DDPG \cite{pmlr-v32-silver14} - Deep Deterministic Policy Gradient also didn't work well with TE problem as it is a continuous problem.
The author after his extensive research and testing purposed two new techniques TE-aware exploration and actor-critic based prioritized experience replay to optimize the general DRL framework.

% Note: for a SEMINAR, Related Work usually is a BAD idea and makes no
% sense! 

\section{Problem Statement}
\label{sec:problem}

A challenging problem which arises in this domain is for performance evaluation and the optimization of the network which is referred to as Traffic Engineering TE problem \cite{Exp:_DRLapproach}. 
The overall goal is to avoid congestion in the network and to boost the total utility function \cite{Exp:_DRLapproach} for all the communication sessions which considers both throughput and delay 
$U\left(x_{k}, z_{k}\right)=U_{\alpha_{1}}\left(x_{k}\right)-\sigma \cdot U_{\alpha_{2}}\left(z_{k}\right)$               

\section{Network Utility Function}
\label{sec:NUM}

The Approach NUM- Network utility function \cite{low_Lapsley:_handb_Flowcontrol}  aims to maximize the overall utility function by allocating resources in the network and thereby providing a solution to the optimization problem. There were certain drawbacks linked with the approach which decreased its efficiency. Firstly the input key factors like user demand, link usages were hard to predict. Secondly, it was hard to estimate the end to end delay without a correct mathematically model. Lastly, NUM approach has failed to address the network dynamics.


\section{Deep Reinforcement Learning (DRL)}
\label{sec:relwork}
The author gave a brief overview of DRL - Deep Reinforcement Learning which is created using Reinforcement Learning and deep learning. 
DRL approach of many fundamental terms like action, state, rewards, Environment policy and value.
The state is the current situation in which the agent/actor finds itself surrounded by objects, tools and obstacles. 
The place or the world in which the agent performs a task is called the environment. The environment consists of the set of rules which processes the actions of the agent and provides the result.
Action - actions are all the possibles task performed by the agent in the environment
Reward - The reward measures the success or the failure of the task performed by the agent. The feedback is provided by the environment for the actions of an agent for any given state. The results are the new state created and rewards if any  

The technological developments have greatly increased the growth in the field of Deep Q Networks which jointly makes use Q learning and deep learning. DQN uses neural network instead of Q learning table to approximate the Q-value function.  

The advance DQN \cite{mnih2015humanlevel} also had many pitfalls like - its performance was not optimal with the increasing sets of action. Plus the exploration strategy used by DQN was not that effective. To counter these policy gradients came into existence which tries to learn more in a robust way by evaluating the action to be taken rather than figuring out the value of each action. Furthermore to even increase its efficiency actor and critic model came into lights in which two sub agents learns together. One learns the policy for action called actor and other studies the Q-value for each action and state called a critic.

\section{Proposed DRL Based Framework}
\label{sec:Proposed_DRL}

DDPG \cite{Exp:_DRLapproach} was chosen as a starting point by the author as it was capable to counter the problem of continuous control which was working fine with a few continuous control tasks but after the experiment done by the author, the results did not offer a decent  performance against the TE problems because of two reasons. Firstly, the method given for physical control problems had some issues. Secondly, ignorance of transition samples as it was capable to deal with the uniform sampling method only. Thus to overcome these shortfalls and to give an optimal solution for TE problem the author planned to include an actor and critic based prioritized experience replay. The author describes the prerequisites as \cite{Exp:_DRLapproach}  which needs to be designed with great care in order to work with DRL model. At First The state space which is formed by two important factors namely throughput and delay for each
communication session. Second An action space which is a collection of split ratios for the different communication session and can be termed as a solution to the TE problem. Lastly, The reward which is basically the utility function of communication sessions which needs to be maximized to counter the TE problem.

This model runs to provide the best action at a time frame by capturing the network state, transition samples and the action to the network.

\section{Algorithm DRL-TE}
\label{sec:relwork}

After the shortfalls of DDPG \cite{pmlr-v32-silver14} method thus to overcome these shortfalls and to give an optimal solution for TE problem the author planned to include an actor and critic based prioritized experience replay  \cite{Exp:_DRLapproach}. 
To derive and experience driven approach which can fully control a dynamic network the author makes use of prioritized experience replay which samples the data on the basis of the priority assigned for each sample in a time frame. The author combined DQL-based DRL method with actor and critic model along with priority experience replay. The algorithm DRL-TE \cite{Exp:_DRLapproach} uses a dual layer loop free connected network for its implementation for actor and critic network.
The First layer consists of 64 neurons, leaky rectifier for activation.
The Second layer consist of an activation function which guarantees that the sum of output values should be one. critic network model also uses the same configuration network as described above.

\section{Implementation of DRL-TE Algorithm}
\label{sec:concl}

The algorithm DRL-TE was implemented using ns-3 simulation environment along with tensor flow for actor and critic networks.
System configuration \cite{Exp:_DRLapproach} was an Intel Quad-core 2.6GHz CPU with 8GB ram. NSFNET, ARPANET and a random typologies were used to evaluate the performance of the algorithm \cite{Exp:_DRLapproach}. Traffic demand followed a Poisson process and the scale was set increasing in each run. Here is the brief summary of the DRL-TE algorithm \cite{Exp:_DRLapproach}.
\begin{itemize}
\item The algorithm in its first step assigns random weight to the actor network $\pi(\cdot)$ and the critic network $Q(\cdot)$ with weights $\theta^{Q}$ and $\theta^{\pi}$ respectively. 
\item Target networks $Q^{\prime}(\cdot)$ and 
$\pi^{\prime}(\cdot)$ with same weight as that of the original network were employed to improved the learning stability.
\item Calculation of key factors for all the samples.
\item  Temporal-Difference (TD) error is calculated for training the  critic network. \\
$\delta_{i}:=y_{i}-Q\left(\mathbf{s}_{i}, \mathbf{a}_{i}\right)$;

TD error is the difference between y which is the target value for training the critic network and the function Q(${s}_{i}$,${a}_{i}$) is the expected return for taking action ${a}_{i}$ while in state ${s}_{i}$.

\item For training of the actor network it is essential to calculate the Q gradient $\nabla_{\boldsymbol{\theta}^{\pi}} J_{i}:=$ \\
$\left.\left.\nabla_{\mathbf{a}} Q(\mathbf{s}, \mathbf{a})\right|_{\mathbf{s}=\mathbf{s}_{i}, \mathbf{a}=\pi}\left(\mathbf{s}_{i}\right) \cdot \nabla_{\boldsymbol{\theta}^{\pi}} \pi(\mathbf{s})\right|_{\mathbf{s}=\mathbf{s}_{i}}$ \\

Actor function is given by $\pi(\mathbf{s})$ and \\
the critic function - $Q(\mathbf{s}, \mathbf{a})$.
To find the Q gradient a chain rule is applied to the expected reward J by the similar research done in    \cite{low_Lapsley:_handb_Flowcontrol}.  
\item TD error and Q gradient both are combined to calculate the priority of the samples. $p_{i}$ \\
$p_{i}:=\varphi \cdot\left(\left|\delta_{i}\right|+\xi\right)+(1-\varphi) \cdot \overline{\left|\nabla_{\mathrm{a}} Q\right|}$ \\

The formula makes uses of factor $\varphi$ which denotes the relative importance of TD Error vs Q Gradients. While $\overline{\left|\nabla_{\mathrm{a}} Q\right|}$ if the average of Q gradients absolute values and $\xi$ used to avoid samples which have negligible samples once.
\item weight changes are accumulated for actor network $\pi(\cdot)$ 
$\Delta_{\boldsymbol{\theta}^{\pi}}:=\Delta_{\boldsymbol{\theta}^{\pi}}+\omega_{i} \cdot \nabla_{\boldsymbol{\theta}^{\pi}} J_{i}$ \\
and critic network $Q(\cdot)$ by  \\
$\Delta_{\boldsymbol{\theta}} \circ:=\Delta_{\boldsymbol{\theta}^{Q}}+\omega_{i} \cdot \delta_{i} \cdot \nabla_{\boldsymbol{\theta}^{Q}} Q\left(\mathbf{s}_{i}, \mathbf{a}_{i}\right)$.
\item The accumulated weight change then used to update the actor and critic network. The final updation also happens in the target network and the rate of updation is defined by a factor $\tau$.
\end{itemize}
The practical implementation of the algorithm DRL-TE is not present in general.

\section{ Performance Evaluation}
\label{sec:pet-peeves}
This experimental approach was then compared with techniques like shortest path , load balance, network utility maximization and DDPG
while keeping all the other factor identical. Some of the conclusions were drawn after evaluation.
\begin{itemize}
\item DRL-TE \cite{Exp:_DRLapproach} Outperforms in terms of reducing end-to-end delay, maximizing total utility function, pretty good end to end throughput on various topologies when compared with other models like SP, LB, NUM and DDPG. 
\item Due to better performance result of DRL-TE \cite{Exp:_DRLapproach} this technique is considered to be robust against the changes in network parameters like traffic load and topology.
\item DRL-TE \cite{Exp:_DRLapproach} proved to be the best by beating the DDPG approach as it was quickly able to reach a state with its own experience which provide a solution with higher reward. On the other hand DDPG was seen stuck with low rewarded solution and thus it was not able to counter DRL-TE. 
\end{itemize}

\section{Discussion}
\label{sec:dis}

Seeing the huge success of AlphaGo by Deepmind the motivation to use Deep reinforcement learning was derived as it was able to handle the complicated states of the AlphaGo. The model developed should be able to learn about the dynamic network patterns and can take decisions when required. Thus the approach was focused to tackle the most fundamental problem of forwarding the data packet from source to destination in a wireless communication network with maximizing the total utility function called Traffic Engineering TE problem. However, comparison has been done with the previously developed approaches like DQN and DDPG Approach. The existing DQN approach was not successful as the TE was a continuous problem and as known DQN was not able to handle the continuous problem. While the Deep deterministic policy gradients (DDPG) approaches implementation was also not successful as the performance result delivered by this approach were also not satisfied as DDPG were not able to process the transition samples because it was based on uniform sampling. However after investigation and research actor and critic model with prioritized experience replay were incorporated into the solutions arsenal. The actor network was inspired by DDPG and can learn the policy in a similar way while the critic network understood the state and action knowledge. 

\section{Conclusion}
\label{sec:concl}

In the end , authors provided an experience-driven DRL-TE  \cite{Exp:_DRLapproach} with actor and critic networks approach to counter TE problem. The performance of this approach when evaluated using NSFNET, APRANET topologies and a random topologies. This approach outperforms all the current baselines methods. As a result of the simulation the DRL-TE approach reduces end to end delay, was robust to network changes, offered better throughput and thus maximizing the total utility function.
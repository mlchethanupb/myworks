\begin{abstract}
Abstract.

(The report has the same structure as the original paper~\cite {Gu}. 
I will write more or less every section from the original paper in my own words).
\end{abstract}

\section{Introduction}%with abstract 1 page
\label{sec:introduction}
(with abstract 1 page)
\begin{enumerate}
\item Introductory example: customized hardware for network functions (firewalls etc.) are not flexible, not scalable and costly -> virtualize them and i.e. rent computing resources.  \\
Lots of new problems: On which server should the VNF be. How mush VNF should be installed.
In the case of service function chains: To which of multiple VNFs should you route the a upcoming flow.
%Consider the following scenario... consider you provide network services

\item Explaining the problem that has been solved in my reference paper~\cite{Gu}: \\
What is orchestration and flow scheduling.\\
Why is it novel: Heuristic solutions have simplified models
 (e.g. they take the end-to-end delay not into consideration), rely on knowledge and are not online.\\
How do they approach it: DRL (problem: large running time, if there is a huge amount of actions) -> DRL with guidance.\\
\end{enumerate}

\section{Model} %1 page
\label{sec:model}
(1 page)
I will NOT write a formal description of the model here (or at least I try it). 
Short overview about the costs that can occur. 
I will try to make some good pictures to explain the costs on example graphs.

\section{Model-Assisted DRL Framework}%1 page
\label{sec:drl}
(1 page)
Also no formal description here. just a explanation/summary.
\begin{enumerate}
\item  Explain the state, action and reward of the DRL.
\item Not making random actions to learn, guide the DRL.
\item Describe their 6 steps of training.
\item Explain the action generation algorithm shortly (just the idea) and how to schedule the algorithms.
\end{enumerate}

\section{Evaluation}% 1/2 page
\label{sec:evaluation}
(1/2 page)
I think I will not use pictures here.
\begin{enumerate}
\item  The setup (but without exact values for the variables -  only what they have done roughly).
\item How good the algorithm performs compared to other algorithms.
\item How to tune the variables to get the best result.
\end{enumerate}



\section{Discussion} %with references 1/2 page, maybe less
\label{sec:relwork}
(with references 1/2 page, maybe less)
\begin {enumerate}
\item This method can be applied to other DRL algorithms.
\end{enumerate}




%
\section{ PERFORMANCE EVALUATION}
\label{sec:pet-peeves}

\begin{itemize}

\item Simulation on NSFNET and ARPANET topologies
\item Results were compared with other methods.
\item Performance DDPG and DRL-TE was evaluated for each network topologies.

\subsection{Factors for comparisons}
\label{sec:orgheadline1}
\begin{itemize}

\item End to end delay
\item End to end throughput.
\item Total Utility.

\end{itemize}

\end{itemize}


 
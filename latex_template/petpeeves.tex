
\section{Pet peeves}
\label{sec:pet-peeves}

\subsection{Language/correctness}
\label{sec:orgheadline1}
\begin{itemize}
\item It is ``related work'' in the singular -- ``work'' is considered a non-countable noun here and hence takes no plural ``s''. Do not think of it as the collection of papers, each one a single ``work''. 
\item Punctuation: Typical punctuation errors occur with compound and complex sentences; in particular, when conjunctive adverbs are used. See here for details: \url{http://www.towson.edu/ows/sentences.htm}
\item Watch out for punctuation rules, in particular, punctuation of defining vs.\ non-defining relative clauses (e.g., ``bla, that'' is almost always wrong).
\item Check hyphenation rules, in particular, for compound attributes (\url{http://en.wikipedia.org/wiki/English_compound#Hyphenated_compound_adjectives}). Roughly: compound nouns tend not to be hyphenated, compound attributes usually are (with plenty of exceptions)
\item ``can not'' and ``cannot'' mean opposite things; often, many people  mean ``cannot'' but incorrectly write ``can not''. E.g., ``This dog can not bark'' (it is able to sometimes shut up) vs.\ ``This dog cannot bark'' (it is unable to produce sound).
\item It is ``et al.'', abbreviating ``et alia''; hence, the period is required after ``al'' and wrong after ``et''
\item Quoting a comma rule from \url{http://grammar.ccc.commnet.edu/grammar/commas.htm} :

  \begin{quote}
    Use a comma to set off introductory elements, as in ``Running
    toward third base, he suddenly realized how stupid he looked.''

    It is permissible to omit the comma after a brief introductory
    element if the omission does not result in confusion or hesitancy
    in reading. If there is ever any doubt, use the comma, as it is
    always correct. 
  \end{quote}
 If you would like some additional guidelines on 
    using a comma after introductory elements, click HERE.
\item Currently, an ``idiot's comma'' is sneaking into common language, offsetting a (perhaps long) subject from its predicate. This is \emph{totally and absurdly} wrong! 
\item The so-called ``zero article'' (leaving out any article from a noun
phrase) is well described here:
\url{http://grammar.ccc.commnet.edu/grammar/determiners/determiners.htm}
(search for ``zero article''); in context with abstract nouns, this
link might be a good start:
\url{http://www.bbc.co.uk/worldservice/learningenglish/youmeus/learnit/learnitv255.shtml}. Incorrect article use really hinders readability. 
\item Whether or not a word or an acronym is preceded by ``a'' or ``an'' depends on whether the initial \emph{sound} is a vowel or a consonant; the written form of word or acronym does not matter!

\item Watch out for correct use of third person singular ``s'' conjugation
\end{itemize}

\subsection{Style}
\label{sec:orgheadline2}
\begin{itemize}
\item Read and follow  \url{http://www.ieee.org/documents/stylemanual.pdf}
\item ``Section 4'', ``Figure 1'', \ldots{} are proper names and are hence capitalized; the word ``section'', ``figure'', \ldots{} WITHOUT the number does not refer to a specific entity, is hence not a proper name, and is hence not capitalized.
\item Avoid ``lonely'' headings -- if there is a 3.1, there must also be a 3.2
\item Use verbs, not adjectives. E.g., ``is dependent on'' $\rightarrow$ $depends on$ (same thing, shorter text)
\item Check rules on capitalization of title case, in particular, for conjuctions, prepositions, etc.
\item Do not use contractions: write ``does not'' instead of ``doesn't''; ``has not'' instead of ``hasn't'', etc.
\end{itemize}

\subsection{Typesetting}
\label{sec:orgheadline3}
\begin{itemize}
\item Use a short space $\backslash,$, between number and unit. It is ``2\,kg'', not ``2 kg'' (and certainly not ``2kg''). Never use ``sec''; the correct abbreviation for second is ``s''. In general, there are plenty of rules for units in the SI system; compare \url{http://physics.nist.gov/cuu/Units/}
\item Units are typeset upright (roman)
\item Multi-letter variables, subscripts etc.\ are typeset upright. Example: \(t_\mathrm{abc}\) would be correct. (only single-letter variables are italic)
\item There is a space between a word and a citation: it is ``bla [1]'',
  not ``bla[1]''. Also, the citation goes BEFORE the full stop, it is
  part of the sentence: ``bla bla [1].'' not: ``bla bla. [1]'' 

\item Distinguish between hyphen and dash ( - or -- in \LaTeX{}) -- this is REALLY annoying. 
\item In many fonts, the signs for opening quotes and closing quotes differ. \LaTeX{} allows the writer to express which ones are desired by ``   for opening quotes and ''  for closing quotes. Using "bla", however, is not correct! Single quotes are set correspondingly.  For languages other than English, there are specific quotation commands as well. (Any decent text preparation systems allows this distinction; consult your manual if you are not using \LaTeX{}.)
\item Correct rules for typesetting math and similar texts are collected
here: \url{http://physics.nist.gov/cuu/pdf/typefaces.pdf}
\url{http://physics.nist.gov/cuu/pdf/sp811.pdf}
\item \textbf{Never, ever} include bitmaps, especially not bitmaps with lossy
compression except for things like photpgrahs. Use vector formats like EMF, EPS, or PDF.
\end{itemize}

\subsection{References}
\label{sec:orgheadline4}
\begin{itemize}
\item BiBTeX: watch out for use of \{\} in title field to ensure correct capitalization
\item BibTeX: mind the warnings of BiBTeX! If BiBTeX complains about a missing field, it usually really is required!
\item When quoting a figure or a table, the caption should state the reference as well as which figure or table it is in the original reference
\end{itemize}

\subsection{Other resources}
\label{sec:orgheadline5}
\begin{itemize}
\item \url{http://www.ece.ucdavis.edu/~jowens/commonerrors.html}
\end{itemize}
